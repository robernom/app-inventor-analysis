\documentclass[a4paper]{article}

\newif\ifdraft
\drafttrue
%\draftfalse

\usepackage{graphicx}
\usepackage{twocolpceurws}
\usepackage{xspace}
\usepackage{color}
\usepackage[usenames,dvipsnames]{xcolor}

\ifdraft
  \newcommand{\grex}[1]{{\color{red}\emph{Gregorio says: #1}}\xspace}
    \newcommand{\rnombela}[1]{{\color{blue}\emph{Roberto says: #1}}\xspace}
  \newcommand{\cn}{\textcolor{pink}{[citation needed]}}
\else
  \usepackage[disable]{todonotes}
  \newcommand{\grex}[1]{}
  \newcommand{\rnombela}[1]{}
  \newcommand{\fixme}[1]{}
  \newcommand{\cn}{}
\fi

\let\labelindent\relax
\usepackage[inline]{enumitem}

\newcommand{\RQ}[1]{\emph{RQ\textsubscript{#1}}}
\newcommand{\HP}[1]{\emph{H\textsubscript{#1}}}
%\newlist{rquestion}{enumerate}{2}
%\setlist[rquestion,1]{label=\RQ{\arabic*.}, itemsep=0cm, topsep=0.1cm, leftmargin=3.4em}
%\setlist[rquestion,2]{label*=\emph{\textsubscript{\arabic*.}}, itemsep=0cm, topsep=0cm, leftmargin=2.6em}

\newcommand{\tbd}{\emph{To be done.}}

\newenvironment{hassanbox}%
{\begin{center}\vspace{1mm}\noindent\begin{Sbox}\begin{minipage}{0.95\columnwidth}}%
{\end{minipage}\end{Sbox}\fbox{\TheSbox}\end{center}\vspace{1mm}}

%%% Local Variables:
%%% mode: latex
%%% TeX-master: "fairness"
%%% End:


\title{An analysis of App Inventor projects}

\author{
Roberto Nombela \\ Universidad Rey Juan Carlos\\
                Madrid, Spain \\ r.nombelaa@alumnos.urjc.es
}

\institution{}

\begin{document}
\maketitle

\begin{abstract}
Visual programming languages help learners to acquire and improve programming and computational skills.
There are many visual programming languages and platforms currently, being Scratch the most popular one.
After Scratch, learners usually move to other visual languages, as for instance MIT App Inventor, a visual programming language block-based to create Android applications.
Many studies exist on the acquisition and assessment of computational thinking skills in Scratch.
However, despite its wide use, similar studies on the use of App Inventor can be seldom found.
That is why in this paper we gather and analyze App Inventor projects with the goal of having an overview of the type of projects that are created.
Our aim in the long run is to create tools that support the learning process of App Inventor.
\end{abstract}


\section{Introduction}

In recent years, the concept of Computational Thinking (CT) has been presented as a process to solve problems, not necessarily only in programming~\cite{wing2006computational}. 
CT includes skills that facilitate the resolution of problems, such as: abstraction, algorithmic and parallel thinking, data representation. These abilities make the CT a fundamental help for any learner.

One way to acquire and improve the skills of CT and programming level are visual programming languages.
This is because programming involves activities such as the ability to design, create and invent new media~\cite{resnick2009scratch}.
The Scratch and App Inventor visual programming langages are examples of this.

Because of this, tools that assess the CT skills from the projects of these programming languages have also emerge, for example Dr. Scratch~\cite{moreno2015dr} and Code Master that analyze the Scratch and App Inventor (and Snap!) projects respectively. 
At current time, these assessmentshave turned out to be more useful for educators than for learners by the fact that the latter do not receive enough feedback to improve (\cn).

\grex{Talk about the goals of the paper}

For the analysis, more than 1000 projects obtained from the Gallery of App Inventor web have been used. 
In this Gallery you can find tens of thousands of projects uploaded by users. The age, gender, nationality or level of programming of the users are unknown.

\grex{Talk about the structure of the paper}

\section{Methodology}


\subsection{Data Gathering}

Using a script the different blocks used by each project are extracted and classified, this information is saved in a csv file. Once we have all projects prepared, we must discard empty (0 blocks) or duplicate projects.

\grex{Tell a little bit more about the data gathering process. How have projects been selected? Have problems been found? Does a list of them exist?}

* COMPARATIVE TABLE: NUMBER DESCRIBE() ORIGINAL, 0 BLOCKS, DUPLICATES *
\grex{include table}

\grex{Say how the data has been extrated? How was the algorithm? How was the rubrique?}

Next, the ``used blocks'' are studied, that is, each type of block adds one for each project in which it has been used. The ``repeated blocks'' are also studied: we add the number of each type of blocks used in each project. We can see that the correlation of both is 0.84, which means that they are highly related.

\section{Results}

Counting the most and least used blocks for all projects we obtain:

* RESULTS *
\grex{include results}

While for projects with a variety less than 10:

* RESULTS *


\subsection{Study of the scores depending on the variety.}

The distribution of the variety of blocks (number of different blocks used) for each project can be seen in the following figure:

\begin{figure}[ht]
\begin{center}
\includegraphics[height=5cm]{fig1}
\caption{Variety distribution }
\label{fig1}
\end{center}
\end{figure}

Projects with more than 10 blocks represent 48.2\% (352 of 730) of the blocks and the scores can be characterized as seen in the following table versus the scores of the projects with less than 10 blocks:

\begin{table}[ht]
\begin{center}
\caption{Comparative table of scores between small and large projects}

\bigskip

\begin{tabular}{|l|l|r|}
\hline
& Large projects & Small projects \\ \hline
Frequency & 352 & 378\\ \hline
Percentage & 48.2 & 51.8\\ \hline
Mean & 18.56 & 9.89\\ \hline
Std. Deviation & 5.33 & 3.27\\ \hline
Minimum & 7 & 3\\ \hline
Maximum & 36 & 19\\ \hline
\end{tabular}
\end{center}
\end{table}


\subsubsection{Acknowledgements}

eMadrid. TBD


\bibliographystyle{alpha} 
\bibliography{samplebib}
%inline the .bbl file directly for mailing to authors.

\end{document}


